\documentclass[a4paper, 12pt, twoside, openright]{article}
\usepackage[T1]{fontenc}
\usepackage[utf8]{inputenc}
\usepackage[frenchb]{babel}
\usepackage{fancyhdr}
\usepackage[final]{graphicx}%@@final
\usepackage{a4wide}
\usepackage{float}
\usepackage{palatino}
\usepackage{rotating}
\usepackage{alltt}
\usepackage{hyperref}
\usepackage{amsmath}
\usepackage{listings}
\usepackage{color}

\definecolor{Ccommentaire}{RGB}{100,135,220}
\definecolor{Cfond}{RGB}{219,246,255}
\definecolor{Ctype}{RGB}{220,0,0}
\definecolor{Cinclude}{RGB}{0,163,0}
\definecolor{Cend}{RGB}{200,0,200}
\definecolor{Cfunc}{RGB}{255,197,0}
\definecolor{Cloop}{RGB}{0,0,255}

\hypersetup{colorlinks,%
citecolor=black,%
filecolor=black,%
linkcolor=black,%
urlcolor=blue}

\pagestyle{fancyplain}
\fancyhead[RE,CE,LE,RO,CO,LO]{}
\fancyfoot[RE,CE,LE,RO,CO,LO]{}
\fancyhead[LE,RO]{\emph{\leftmark}}
\fancyfoot[LE,RO]{Page \thepage}

\renewcommand{\plainfootrulewidth}{0.4pt}
\renewcommand{\footrulewidth}{0.4pt}
\renewcommand{\headrulewidth}{0.4pt}
\renewcommand{\plainheadrulewidth}{0.4pt}

\setlength{\headheight}{20pt} %otherwise there are warnings
\setlength{\parskip}{8pt}

\widowpenalty=10000
\clubpenalty=10000
\raggedbottom

\newcommand{\img}[3][15]{
\begin{figure}[H]
\begin{center}\includegraphics[width= #1 cm]{#2}\end{center}
\caption{#3}
\label{#2}
\end{figure}
}

\newcommand{\code}[2]{
	\lstinputlisting[title = {#2}]{#1}
}


\newcommand{\centrer}[1]{\begin{center}#1\end{center}}

\lstset{tabsize=4,
firstnumber=0, numbers=left, numberstyle=\tiny, numbersep=5pt, stepnumber=5,
breaklines=true, extendedchars=false,
frame=single, backgroundcolor=\color{Cfond},framerule=1pt,
	classoffset=0,morekeywords={int, char, long, short, unsigned, void},keywordstyle=\color{Ctype},
	classoffset=1,morekeywords={typedef, struct},keywordstyle=\textbf,
	classoffset=2,morekeywords={for, while, do},keywordstyle=\color{Cloop},
	classoffset=3,morekeywords={return, exit},keywordstyle=\color{Cend},
	classoffset=4,morekeywords={printf, fprintf, sprintf, snprintf, scanf, fscanf},keywordstyle=\color{Cfunc},
	classoffset=5,
	morecomment=[l][\color{Cinclude}]{\#include},
	morecomment=[l][\color{Cinclude}]{\#define},
	morecomment=[s][\color{Ccommentaire}]{/*}{*/}
}

\author {trax Omar \bsc{Givernaud}}
\title{How to bebin an open source software development project}


\begin{document}
\maketitle
\thispagestyle{empty}
\clearpage
\tableofcontents{}
\thispagestyle{empty}
\clearpage

  \section{Tools}

  \subsection{Mandatory}
  \subsubsection{Source Content Management}

  \paragraph{Why}

  A good scm allow you to: 
  \begin{itemize}
    \item Save your work and rollback easily
    \item Keep trace of what you did, and how.
    \item Implement features easily: branch, 
      so you don't break your project during development, and you can still commit.
    \item Share, and work together.
    \item 
  \end{itemize}

  <h6>What</h6>

  \begin{itemize}\item git or GTFO\end{itemize}

  \subsubsection{Tickets}

  \paragraph{Why}

  \begin{itemize}
    \item For bug reports.
    \item For feature request.
    \item As a todo list/backlog.
  \end{itemize}

  \paragraph{What}
  \begin{itemize}
    \item Redmine/chiliproject
    \item Jira
    \item Avoid trac if your project contains subprojects
  \end{itemize}


  \subsubsection{Unit test framework}
  \paragraph{Why}

  Unit test are awesome for:
  \begin{itemize}
    \item Debugging.
    \item Avoid regressions.
    \item They give nice examples of how to use your code.
  \end{itemize}
  

  \subsubsection{Packager}

  \paragraph{Why}

  If you want your project to work you need to make it easy to test. 
  One way is to package it nicely. 

  \paragraph{What}
  If your whatever language, framework, does not provide it:
  \begin{itemize}
    \item cpack (which can be used without cmake)
    \item fpm
    \item Be a man and right your own rules/specs/ebuild
  \end{itemize}


  \subsection{Nice to have}

  \subsubsection{Continuous integration automate}

  \paragraph{Why}

  \begin{itemize}
    \item It nice that it works on your computer, but it is better if it works on main targets (provides a packages for each target).
    \item Runs all the tools bellow in background.
  \end{itemize}
  

  \paragraph{What}
  \begin{itemize}
    \item Jenkins
    \item gitlab-ci
    \item buildbot
  \end{itemize}


  \subsubsection{Static analyser }
  \paragraph{Why}
  
  Efficient quality checking and nice advice at a low price. What's else?

  \paragraph{What}

  \begin{itemize}
    \item clang/gcc
    \item cppcheck/clint
    \item sonar
  \end{itemize}

  \subsubsection{Runtime analyser}

  

  \subsubsection{I18n}
  \subsubsection{Website}
  \subsubsection{Documentation tools}
  


\end{document}